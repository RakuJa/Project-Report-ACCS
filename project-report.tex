% 	INSPIRED AND MADE FROM:        
        
        %%******************************************%%
        %%                                          %%
        %%        Modello di tesi di laurea         %%
        %%            di Andrea Giraldin            %%
        %%                                          %%
        %%             2 novembre 2012              %%
        %%                                          %%
        %%******************************************%%


% I seguenti commenti speciali impostano:
% 1. 
% 2. PDFLaTeX come motore di composizione;
% 3. project-report.tex come documento principale;
% 4. il controllo ortografico italiano per l'editor.

% !TEX encoding = UTF-8
% !TEX TS-program = pdflatex
% !TEX root = project-report.tex
% !TEX spellcheck = it-IT

% PDF/A filecontents
\RequirePackage{filecontents}
\begin{filecontents*}{\jobname.xmpdata}
  \Title{Applied cryptography Cloud-Storage project report}
  \Author{Daniele Giachetto}
  \Language{it-IT}
  \Subject{This document describes project specifications, design choices and format of all the exchanged messages, using sequence diagrams of every used communication protocol.}
  \Keywords{Applied cryptography\sep Openssl\sep Pisa}
\end{filecontents*}

\documentclass[10pt,                    % corpo del font principale
               a4paper,                 % carta A4
               twoside,                 % impagina per fronte-retro
               openright,               % inizio capitoli a destra
               english,                 
               italian,                 
               ]{book}    

%**************************************************************
% Importazione package
%************************************************************** 

\PassOptionsToPackage{dvipsnames}{xcolor} % colori PDF/A

\usepackage{colorprofiles}

\usepackage[a-2b,mathxmp]{pdfx}[2018/12/22]
                                        % configurazione PDF/A
                                        % validare in https://www.pdf-online.com/osa/validate.aspx

%\usepackage{amsmath,amssymb,amsthm}    % matematica

\usepackage[T1]{fontenc}                % codifica dei font:
                                        % NOTA BENE! richiede una distribuzione *completa* di LaTeX

\usepackage[utf8]{inputenc}             % codifica di input; anche [latin1] va bene
                                        % NOTA BENE! va accordata con le preferenze dell'editor

\usepackage[italian]{babel}    % per scrivere in italiano e in inglese;
                                        % l'ultima lingua (l'italiano) risulta predefinita

\usepackage{bookmark}                   % segnalibri

\usepackage{caption}                    % didascalie

\usepackage{chngpage,calc}              % centra il frontespizio

\usepackage{csquotes}                   % gestisce automaticamente i caratteri (")

\usepackage{emptypage}                  % pagine vuote senza testatina e piede di pagina

\usepackage{epigraph}			% per epigrafi

\usepackage{eurosym}                    % simbolo dell'euro

%\usepackage{indentfirst}               % rientra il primo paragrafo di ogni sezione

\usepackage{graphicx}                   % immagini

\usepackage{hyperref}                   % collegamenti ipertestuali

\usepackage[binding=5mm]{layaureo}      % margini ottimizzati per l'A4; rilegatura di 5 mm

\usepackage{listings}                   % codici

\usepackage{microtype}                  % microtipografia

\usepackage{mparhack,fixltx2e,relsize}  % finezze tipografiche

\usepackage{nameref}                    % visualizza nome dei riferimenti                                      
\usepackage[font=small]{quoting}        % citazioni

\usepackage{subfig}                     % sottofigure, sottotabelle

\usepackage[italian]{varioref}          % riferimenti completi della pagina

\usepackage{booktabs}                   % tabelle                                       
\usepackage{tabularx}                   % tabelle di larghezza prefissata                                    
\usepackage{longtable}                  % tabelle su più pagine                                        
\usepackage{ltxtable}                   % tabelle su più pagine e adattabili in larghezza

\usepackage[toc, acronym]{glossaries}   % glossario
                                        % per includerlo nel documento bisogna:
                                        % 1. compilare una prima volta project-report.tex;
                                        % 2. eseguire: makeindex -s project-report.ist -t project-report.glg -o project-report.gls project-report.glo
                                        % 3. eseguire: makeindex -s project-report.ist -t project-report.alg -o project-report.acr project-report.acn
                                        % 4. compilare due volte project-report.tex.

\usepackage[backend=biber,style=verbose-ibid,hyperref,backref]{biblatex}
                                        % eccellente pacchetto per la bibliografia; 
                                        % produce uno stile di citazione autore-anno; 
                                        % lo stile "numeric-comp" produce riferimenti numerici
                                        % per includerlo nel documento bisogna:
                                        % 1. compilare una prima volta project-report.tex;
                                        % 2. eseguire: biber project-report
                                        % 3. compilare ancora project-report.tex.

\input{project-report-config}                     % file con le impostazioni personali

\begin{document}
%**************************************************************
% Materiale iniziale
%**************************************************************
\frontmatter
\input{inizio-fine/frontespizio}
% \input{inizio-fine/sommario}
\input{inizio-fine/indici}
\cleardoublepage

%**************************************************************
% Materiale principale
%**************************************************************
\mainmatter
% !TEX encoding = UTF-8
% !TEX TS-program = pdflatex
% !TEX root = ../tesi.tex

%**************************************************************
\chapter{Introduzione}
\label{cap:introduzione}
%**************************************************************

%**************************************************************
\section{Project goal}

The students must implement a Client-Server application that resembles a Cloud Storage. Each user has a  ``dedicated storage'' on the server, and User A cannot access User B ``dedicated storage''. Users can Upload, Download, Rename, or Delete data to/from the Cloud Storage in a safe manner.
\begin{itemize}
	\item \textbf{Upload}: the user will upload a file found on the client machine to the server. The uploaded file size can be up to 4GB. The file will be saved with its original name, if this is not possible the file will not be uploaded;
	\item \textbf{Download}: the user will download a file from the server to the client machine. The file will be saved with its original name, if this is not possible the file will not be downloaded; 
	\item \textbf{Delete}: the user specifies a file on his ``dedicated storage'', the user will also be prompted with a choice to confirm the delete operation. The server deletes the file if it was found; 
	\item \textbf{List}: the user asks to the server the list of the filenames that are available on his ``dedicated storage''. The client shows the result to screen;
	\item \textbf{Rename}: the user specifies two filenames, the first one of the file that should be renamed and the second one the new filename desired. If the renaming operation is not possible, the filename is not changed;
\end{itemize}

Once connected to the service, the client can grafecully close the connection with the server.
\newpage{}

%**************************************************************
\section{Assumptions \& Requisites}
Users and server will work under some assumptions that will be described in detail here: \newline{}
\textbf{Users}: 
\begin{itemize}
	\item they are pre-registered on the server;
	\item they have already the CA certificate;
	\item they have each a long-term RSA key-pair;
	\item the long-term private key is password-protected.
\end{itemize}
\textbf{Server}:
\begin{itemize}
	\item it has its own certificate signed by the CA;
	\item it knows the username of every registered user;
	\item it knows the RSA public key of every user;
	\item ``dedicated storage'' is already allocated.
\end{itemize}
When the client application starts, Server and Client must authenticate: \newline{}
\begin{itemize}
	\item server must authenticate with the public key certified by the certification authority;
	\item client must authenticate with the public key, pre-shared with the server.
\end{itemize}

During authentication a symmetric session key must be negotiated: \newline{}
\begin{itemize}
	\item the negotiation must provide Perfect Forward Secrecy;
	\item the entire session must be encrypted and authenticated;
	\item the entire session must be protected against replay attacks.
	
\end{itemize}
%**************************************************************

\section{Technology stack}

The teck stack will be listed in the Table \ref{tab:teck-stack}.
\begin{longtable}{|p{0.2\textwidth}|p{0.8\textwidth}|}
	\caption{Technology stack}
	\label{Technology stack} \\
	\hline
	\textbf{Name} & \textbf{Description} \\
	\hline
	ciao & mamma \\
	\hline
	due & tre \\
	\hline
\end{longtable}%
	

             % Introduction
% !TEX encoding = UTF-8
% !TEX TS-program = pdflatex
% !TEX root = ../tesi.tex

%**************************************************************
\chapter{Handshake \& Session key}
\label{cap:handshake-session-key}
%**************************************************************

%**************************************************************
\section{How the handshake is structured}

The handshake protocol is used to exchange a session key to encrypt and authenticate all the messages after the handshake. It guarantees perfect forward secrecy and usese nonces both client and server side to deny possible replays attack. The handshake is made of 3 steps that are summarized in the following figure: \ref{fig:handshake}.

\begin{figure}[!h] 
    \centering 
    \includegraphics[width=1\columnwidth]{handshake.png} 
    \caption{Handshake.}
    \label{fig:handshake}
\end{figure}

\begin{enumerate}
	\item client sends a random number also called nonce and the chosen username;
	\item server checks validity of username and prepares the preliminary steps to generate a shared session key:
	\begin{itemize}
		\item checks if the username does not contain invalid characters, only alphanumeric and dot characters (.) strictly not in the first position are allowed. If the username is not valid connection is closed allerting the client;
		\item checks if the username is registered, if not then the connection gets closed by sending an error message to the client;
		\item choses a random number \(\alpha\) and uses it to calculate the public Diffie-Hellman key that we will call ``A'' as A = \(g^\alpha mod p \);
		\item the server now sends his certificate, a random number and ``A''. It also signs, using the server private RSA key, the random number received from the client Nonce(c) and A together and sends the signature to the client.
	\end{itemize}
	\item the client checks validity of certificate and signature and generates the shared session key:
	\begin{itemize}
		\item validates server certificate using the CA;
		\item uses server public key to validate the signature received;
		\item choses a random number \(\beta\) and uses it to calculate the public Diffie-Hellman key that we will call ``B'' as B = \(g^\beta mod p \);
		\item generates shared secret Kab = \(g^{\alpha*B} mod p\);
		\item generates shared key hashing Kab with SHA256 and truncating the result to 128 bits;
		\item signs, using the client private RSA key, the random received Nonce(s) and ``B'';
		\item eliminates ``A'', ``B'', \(\alpha\);
		\item sends the generated signature. 
	\end{itemize}
	\item the server checks validity of signature and generates the shared session key:
	\begin{itemize}
		\item uses client public key to validate the signature received;
		\item generates shared secret Kab = \(g^{\beta*A} mod p\);
		\item generates shared key hashing Kab with SHA256 and truncating the result to 128 bits;
		\item eliminates ``A'', ``B'', \(\beta\);
	\end{itemize}	 
\end{enumerate}

%**************************************************************
\section{How is the session key used}

%**************************************************************             % Handshake
%\input{capitoli/capitolo-3}             % Strumenti e tecnologie
%\input{capitoli/capitolo-4}             % Progettazione e sviluppo
%\input{capitoli/capitolo-5}             % Verifica e validazione
%\input{capitoli/capitolo-6}             % Conclusioni
\appendix                               
%\input{capitoli/capitolo-A}             % Appendice A

%**************************************************************
% Materiale finale
%**************************************************************
\backmatter
\printglossary[title={Glossario}]
% \input{inizio-fine/bibliografia}
\end{document}
