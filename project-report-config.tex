%**************************************************************
% file contenente le impostazioni del report
%**************************************************************

%**************************************************************
% Frontespizio
%**************************************************************

% Autore
\newcommand{\myName}{Daniele Giachetto}                                    
\newcommand{\myTitle}{Applied cryptography Cloud-Storage}

% Tipo di progetto                   
\newcommand{\myDegree}{Project report}

% Università             
\newcommand{\myUni}{Università degli studi di Pisa}

% Facoltà       
\newcommand{\myFaculty}{M.Sc. Cybersecurity}

% Dipartimento
\newcommand{\myDepartment}{Department of Information Engineering}

% Titolo del docente
\newcommand{\profTitle}{Prof.}

% Docente
\newcommand{\myProf}{Michele La Manna}

% Luogo
\newcommand{\myLocation}{Pisa}

% Anno accademico
\newcommand{\myAA}{2021-2022}

% Data consegna
\newcommand{\myTime}{June 2022}


%**************************************************************
% Impostazioni di impaginazione
% see: http://wwwcdf.pd.infn.it/AppuntiLinux/a2547.htm
%**************************************************************

\setlength{\parindent}{14pt}   % larghezza rientro della prima riga
\setlength{\parskip}{0pt}   % distanza tra i paragrafi


%**************************************************************
% Impostazioni di biblatex
%**************************************************************
\bibliography{bibliografia} % database di biblatex 

\defbibheading{bibliography} {
    \cleardoublepage
    \phantomsection 
    \addcontentsline{toc}{chapter}{\bibname}
    \chapter*{\bibname\markboth{\bibname}{\bibname}}
}

\setlength\bibitemsep{1.5\itemsep} % spazio tra entry

\DeclareBibliographyCategory{opere}
\DeclareBibliographyCategory{web}

\addtocategory{opere}{womak:lean-thinking}
\addtocategory{web}{site:agile-manifesto}

\defbibheading{opere}{\section*{Riferimenti bibliografici}}
\defbibheading{web}{\section*{Siti Web consultati}}


%**************************************************************
% Impostazioni di caption
%**************************************************************
\captionsetup{
    tableposition=top,
    figureposition=bottom,
    font=small,
    format=hang,
    labelfont=bf
}

%**************************************************************
% Impostazioni di glossaries
%**************************************************************
%**************************************************************
% Glossario
%**************************************************************
\renewcommand{\glossaryname}{Glossario}

\newglossaryentry{Tor}
{
    name=\glslink{Tor}{TOR},
    text=Tor,
    sort=tor,
    description={The Onion Router è un software libero, rilasciato su licenza BSD, con un'interfaccia di gestione disponibile (Vidalia), che permette una comunicazione anonima per Internet basata sulla seconda generazione del protocollo di rete di onion routing: tramite il suo utilizzo è molto più difficile tracciare l'attività Internet dell'utente essendo finalizzato a proteggere la privacy degli utenti, la loro libertà e la possibilità di condurre delle comunicazioni confidenziali senza che vengano monitorate o intercettate.}
}


 % database di termini
\makeglossaries


%**************************************************************
% Impostazioni di graphicx
%**************************************************************
\graphicspath{{immagini/}} % cartella dove sono riposte le immagini


%**************************************************************
% Impostazioni di hyperref
%**************************************************************
\hypersetup{
    %hyperfootnotes=false,
    %pdfpagelabels,
    %draft,	% = elimina tutti i link (utile per stampe in bianco e nero)
    colorlinks=true,
    linktocpage=true,
    pdfstartpage=1,
    pdfstartview=,
    % decommenta la riga seguente per avere link in nero (per esempio per la stampa in bianco e nero)
    %colorlinks=false, linktocpage=false, pdfborder={0 0 0}, pdfstartpage=1, pdfstartview=FitV,
    breaklinks=true,
    pdfpagemode=UseNone,
    pageanchor=true,
    pdfpagemode=UseOutlines,
    plainpages=false,
    bookmarksnumbered,
    bookmarksopen=true,
    bookmarksopenlevel=1,
    hypertexnames=true,
    pdfhighlight=/O,
    %nesting=true,
    %frenchlinks,
    urlcolor=webbrown,
    linkcolor=RoyalBlue,
    citecolor=webgreen,
    %pagecolor=RoyalBlue,
    %urlcolor=Black, linkcolor=Black, citecolor=Black, %pagecolor=Black,
    pdftitle={\myTitle},
    pdfauthor={\textcopyright\ \myName, \myUni, \myFaculty},
    pdfsubject={},
    pdfkeywords={},
    pdfcreator={pdfLaTeX},
    pdfproducer={LaTeX}
}

%**************************************************************
% Impostazioni di itemize
%**************************************************************
\renewcommand{\labelitemi}{$\ast$}

%\renewcommand{\labelitemi}{$\bullet$}
%\renewcommand{\labelitemii}{$\cdot$}
%\renewcommand{\labelitemiii}{$\diamond$}
%\renewcommand{\labelitemiv}{$\ast$}


%**************************************************************
% Impostazioni di listings
%**************************************************************
\renewcommand{\lstlistingname}{Listato}
\lstset{
    language=[LaTeX]Tex,%C++,
    keywordstyle=\color{RoyalBlue}, %\bfseries,
    basicstyle=\small\ttfamily,
    language=Python,
    %identifierstyle=\color{NavyBlue},
    commentstyle=\color{Green}\ttfamily,
    stringstyle=\rmfamily,
    numbers=none, %left,%
    numberstyle=\scriptsize, %\tiny
    stepnumber=5,
    numbersep=8pt,
    showstringspaces=false,
    breaklines=true,
    frameround=ftff,
    frame=single
} 


%**************************************************************
% Impostazioni di xcolor
%**************************************************************
\definecolor{webgreen}{rgb}{0,.5,0}
\definecolor{webbrown}{rgb}{.6,0,0}


%**************************************************************
% Altro
%**************************************************************

\newcommand{\omissis}{[\dots\negthinspace]} % produce [...]

% eccezioni all'algoritmo di sillabazione
\hyphenation
{
    ma-cro-istru-zio-ne
    gi-ral-din
}

%\newcommand{\sectionname}{sezione}
%\addto\captionsitalian{\renewcommand{\figurename}{Figura}
                    %   \renewcommand{\tablename}{Tabella}}

\newcommand{\glsfirstoccur}{\ap{{[g]}}}

\newcommand{\intro}[1]{\emph{\textsf{#1}}}

%**************************************************************
% Environment per ``rischi''
%**************************************************************
\newcounter{riskcounter}                % define a counter
\setcounter{riskcounter}{0}             % set the counter to some initial value

%%%% Parameters
% #1: Title
\newenvironment{risk}[1]{
    \refstepcounter{riskcounter}        % increment counter
    \par \noindent                      % start new paragraph
    \textbf{\arabic{riskcounter}. #1}   % display the title before the 
                                        % content of the environment is displayed 
}{
    \par\medskip
}

\newcommand{\riskname}{Rischio}

\newcommand{\riskdescription}[1]{\textbf{\\Descrizione:} #1.}

\newcommand{\risksolution}[1]{\textbf{\\Soluzione:} #1.}

%**************************************************************
% Environment per ``use case''
%**************************************************************
\newcounter{usecasecounter}             % define a counter
\setcounter{usecasecounter}{0}          % set the counter to some initial value

%%%% Parameters
% #1: ID
% #2: Nome
\newenvironment{usecase}[2]{
    \renewcommand{\theusecasecounter}{\usecasename #1}  % this is where the display of 
                                                        % the counter is overwritten/modified
    \refstepcounter{usecasecounter}             % increment counter
    \vspace{10pt}
    \par \noindent                              % start new paragraph
    {\large \textbf{\usecasename #1: #2}}       % display the title before the 
                                                % content of the environment is displayed 
    \medskip
}{
    \medskip
}

\newcommand{\usecasename}{UC}

\newcommand{\usecaseactors}[1]{\textbf{\\Attori Principali:} #1. \vspace{4pt}}
\newcommand{\usecasepre}[1]{\textbf{\\Precondizioni:} #1. \vspace{4pt}}
\newcommand{\usecasedesc}[1]{\textbf{\\Descrizione:} #1. \vspace{4pt}}
\newcommand{\usecasepost}[1]{\textbf{\\Postcondizioni:} #1. \vspace{4pt}}
\newcommand{\usecasealt}[1]{\textbf{\\Scenario Alternativo:} #1. \vspace{4pt}}

%**************************************************************
% Environment per ``namespace description''
%**************************************************************

\newenvironment{namespacedesc}{
    \vspace{10pt}
    \par \noindent                              % start new paragraph
    \begin{description} 
}{
    \end{description}
    \medskip
}

\newcommand{\classdesc}[2]{\item[\textbf{#1:}] #2}
